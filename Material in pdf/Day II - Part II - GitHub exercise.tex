\documentclass[11pt]{article}

    \usepackage[breakable]{tcolorbox}
    \usepackage{parskip} % Stop auto-indenting (to mimic markdown behaviour)
    
    \usepackage{iftex}
    \ifPDFTeX
    	\usepackage[T1]{fontenc}
    	\usepackage{mathpazo}
    \else
    	\usepackage{fontspec}
    \fi

    % Basic figure setup, for now with no caption control since it's done
    % automatically by Pandoc (which extracts ![](path) syntax from Markdown).
    \usepackage{graphicx}
    % Maintain compatibility with old templates. Remove in nbconvert 6.0
    \let\Oldincludegraphics\includegraphics
    % Ensure that by default, figures have no caption (until we provide a
    % proper Figure object with a Caption API and a way to capture that
    % in the conversion process - todo).
    \usepackage{caption}
    \DeclareCaptionFormat{nocaption}{}
    \captionsetup{format=nocaption,aboveskip=0pt,belowskip=0pt}

    \usepackage[Export]{adjustbox} % Used to constrain images to a maximum size
    \adjustboxset{max size={0.9\linewidth}{0.9\paperheight}}
    \usepackage{float}
    \floatplacement{figure}{H} % forces figures to be placed at the correct location
    \usepackage{xcolor} % Allow colors to be defined
    \usepackage{enumerate} % Needed for markdown enumerations to work
    \usepackage{geometry} % Used to adjust the document margins
    \usepackage{amsmath} % Equations
    \usepackage{amssymb} % Equations
    \usepackage{textcomp} % defines textquotesingle
    % Hack from http://tex.stackexchange.com/a/47451/13684:
    \AtBeginDocument{%
        \def\PYZsq{\textquotesingle}% Upright quotes in Pygmentized code
    }
    \usepackage{upquote} % Upright quotes for verbatim code
    \usepackage{eurosym} % defines \euro
    \usepackage[mathletters]{ucs} % Extended unicode (utf-8) support
    \usepackage{fancyvrb} % verbatim replacement that allows latex
    \usepackage{grffile} % extends the file name processing of package graphics 
                         % to support a larger range
    \makeatletter % fix for grffile with XeLaTeX
    \def\Gread@@xetex#1{%
      \IfFileExists{"\Gin@base".bb}%
      {\Gread@eps{\Gin@base.bb}}%
      {\Gread@@xetex@aux#1}%
    }
    \makeatother

    % The hyperref package gives us a pdf with properly built
    % internal navigation ('pdf bookmarks' for the table of contents,
    % internal cross-reference links, web links for URLs, etc.)
    \usepackage{hyperref}
    % The default LaTeX title has an obnoxious amount of whitespace. By default,
    % titling removes some of it. It also provides customization options.
    \usepackage{titling}
    \usepackage{longtable} % longtable support required by pandoc >1.10
    \usepackage{booktabs}  % table support for pandoc > 1.12.2
    \usepackage[inline]{enumitem} % IRkernel/repr support (it uses the enumerate* environment)
    \usepackage[normalem]{ulem} % ulem is needed to support strikethroughs (\sout)
                                % normalem makes italics be italics, not underlines
    \usepackage{mathrsfs}
    

    
    % Colors for the hyperref package
    \definecolor{urlcolor}{rgb}{0,.145,.698}
    \definecolor{linkcolor}{rgb}{.71,0.21,0.01}
    \definecolor{citecolor}{rgb}{.12,.54,.11}

    % ANSI colors
    \definecolor{ansi-black}{HTML}{3E424D}
    \definecolor{ansi-black-intense}{HTML}{282C36}
    \definecolor{ansi-red}{HTML}{E75C58}
    \definecolor{ansi-red-intense}{HTML}{B22B31}
    \definecolor{ansi-green}{HTML}{00A250}
    \definecolor{ansi-green-intense}{HTML}{007427}
    \definecolor{ansi-yellow}{HTML}{DDB62B}
    \definecolor{ansi-yellow-intense}{HTML}{B27D12}
    \definecolor{ansi-blue}{HTML}{208FFB}
    \definecolor{ansi-blue-intense}{HTML}{0065CA}
    \definecolor{ansi-magenta}{HTML}{D160C4}
    \definecolor{ansi-magenta-intense}{HTML}{A03196}
    \definecolor{ansi-cyan}{HTML}{60C6C8}
    \definecolor{ansi-cyan-intense}{HTML}{258F8F}
    \definecolor{ansi-white}{HTML}{C5C1B4}
    \definecolor{ansi-white-intense}{HTML}{A1A6B2}
    \definecolor{ansi-default-inverse-fg}{HTML}{FFFFFF}
    \definecolor{ansi-default-inverse-bg}{HTML}{000000}

    % commands and environments needed by pandoc snippets
    % extracted from the output of `pandoc -s`
    \providecommand{\tightlist}{%
      \setlength{\itemsep}{0pt}\setlength{\parskip}{0pt}}
    \DefineVerbatimEnvironment{Highlighting}{Verbatim}{commandchars=\\\{\}}
    % Add ',fontsize=\small' for more characters per line
    \newenvironment{Shaded}{}{}
    \newcommand{\KeywordTok}[1]{\textcolor[rgb]{0.00,0.44,0.13}{\textbf{{#1}}}}
    \newcommand{\DataTypeTok}[1]{\textcolor[rgb]{0.56,0.13,0.00}{{#1}}}
    \newcommand{\DecValTok}[1]{\textcolor[rgb]{0.25,0.63,0.44}{{#1}}}
    \newcommand{\BaseNTok}[1]{\textcolor[rgb]{0.25,0.63,0.44}{{#1}}}
    \newcommand{\FloatTok}[1]{\textcolor[rgb]{0.25,0.63,0.44}{{#1}}}
    \newcommand{\CharTok}[1]{\textcolor[rgb]{0.25,0.44,0.63}{{#1}}}
    \newcommand{\StringTok}[1]{\textcolor[rgb]{0.25,0.44,0.63}{{#1}}}
    \newcommand{\CommentTok}[1]{\textcolor[rgb]{0.38,0.63,0.69}{\textit{{#1}}}}
    \newcommand{\OtherTok}[1]{\textcolor[rgb]{0.00,0.44,0.13}{{#1}}}
    \newcommand{\AlertTok}[1]{\textcolor[rgb]{1.00,0.00,0.00}{\textbf{{#1}}}}
    \newcommand{\FunctionTok}[1]{\textcolor[rgb]{0.02,0.16,0.49}{{#1}}}
    \newcommand{\RegionMarkerTok}[1]{{#1}}
    \newcommand{\ErrorTok}[1]{\textcolor[rgb]{1.00,0.00,0.00}{\textbf{{#1}}}}
    \newcommand{\NormalTok}[1]{{#1}}
    
    % Additional commands for more recent versions of Pandoc
    \newcommand{\ConstantTok}[1]{\textcolor[rgb]{0.53,0.00,0.00}{{#1}}}
    \newcommand{\SpecialCharTok}[1]{\textcolor[rgb]{0.25,0.44,0.63}{{#1}}}
    \newcommand{\VerbatimStringTok}[1]{\textcolor[rgb]{0.25,0.44,0.63}{{#1}}}
    \newcommand{\SpecialStringTok}[1]{\textcolor[rgb]{0.73,0.40,0.53}{{#1}}}
    \newcommand{\ImportTok}[1]{{#1}}
    \newcommand{\DocumentationTok}[1]{\textcolor[rgb]{0.73,0.13,0.13}{\textit{{#1}}}}
    \newcommand{\AnnotationTok}[1]{\textcolor[rgb]{0.38,0.63,0.69}{\textbf{\textit{{#1}}}}}
    \newcommand{\CommentVarTok}[1]{\textcolor[rgb]{0.38,0.63,0.69}{\textbf{\textit{{#1}}}}}
    \newcommand{\VariableTok}[1]{\textcolor[rgb]{0.10,0.09,0.49}{{#1}}}
    \newcommand{\ControlFlowTok}[1]{\textcolor[rgb]{0.00,0.44,0.13}{\textbf{{#1}}}}
    \newcommand{\OperatorTok}[1]{\textcolor[rgb]{0.40,0.40,0.40}{{#1}}}
    \newcommand{\BuiltInTok}[1]{{#1}}
    \newcommand{\ExtensionTok}[1]{{#1}}
    \newcommand{\PreprocessorTok}[1]{\textcolor[rgb]{0.74,0.48,0.00}{{#1}}}
    \newcommand{\AttributeTok}[1]{\textcolor[rgb]{0.49,0.56,0.16}{{#1}}}
    \newcommand{\InformationTok}[1]{\textcolor[rgb]{0.38,0.63,0.69}{\textbf{\textit{{#1}}}}}
    \newcommand{\WarningTok}[1]{\textcolor[rgb]{0.38,0.63,0.69}{\textbf{\textit{{#1}}}}}
    
    
    % Define a nice break command that doesn't care if a line doesn't already
    % exist.
    \def\br{\hspace*{\fill} \\* }
    % Math Jax compatibility definitions
    \def\gt{>}
    \def\lt{<}
    \let\Oldtex\TeX
    \let\Oldlatex\LaTeX
    \renewcommand{\TeX}{\textrm{\Oldtex}}
    \renewcommand{\LaTeX}{\textrm{\Oldlatex}}
    % Document parameters
    % Document title
    \title{Day II - Part II - GitHub exercise}
    
    
    
    
    
% Pygments definitions
\makeatletter
\def\PY@reset{\let\PY@it=\relax \let\PY@bf=\relax%
    \let\PY@ul=\relax \let\PY@tc=\relax%
    \let\PY@bc=\relax \let\PY@ff=\relax}
\def\PY@tok#1{\csname PY@tok@#1\endcsname}
\def\PY@toks#1+{\ifx\relax#1\empty\else%
    \PY@tok{#1}\expandafter\PY@toks\fi}
\def\PY@do#1{\PY@bc{\PY@tc{\PY@ul{%
    \PY@it{\PY@bf{\PY@ff{#1}}}}}}}
\def\PY#1#2{\PY@reset\PY@toks#1+\relax+\PY@do{#2}}

\expandafter\def\csname PY@tok@w\endcsname{\def\PY@tc##1{\textcolor[rgb]{0.73,0.73,0.73}{##1}}}
\expandafter\def\csname PY@tok@c\endcsname{\let\PY@it=\textit\def\PY@tc##1{\textcolor[rgb]{0.25,0.50,0.50}{##1}}}
\expandafter\def\csname PY@tok@cp\endcsname{\def\PY@tc##1{\textcolor[rgb]{0.74,0.48,0.00}{##1}}}
\expandafter\def\csname PY@tok@k\endcsname{\let\PY@bf=\textbf\def\PY@tc##1{\textcolor[rgb]{0.00,0.50,0.00}{##1}}}
\expandafter\def\csname PY@tok@kp\endcsname{\def\PY@tc##1{\textcolor[rgb]{0.00,0.50,0.00}{##1}}}
\expandafter\def\csname PY@tok@kt\endcsname{\def\PY@tc##1{\textcolor[rgb]{0.69,0.00,0.25}{##1}}}
\expandafter\def\csname PY@tok@o\endcsname{\def\PY@tc##1{\textcolor[rgb]{0.40,0.40,0.40}{##1}}}
\expandafter\def\csname PY@tok@ow\endcsname{\let\PY@bf=\textbf\def\PY@tc##1{\textcolor[rgb]{0.67,0.13,1.00}{##1}}}
\expandafter\def\csname PY@tok@nb\endcsname{\def\PY@tc##1{\textcolor[rgb]{0.00,0.50,0.00}{##1}}}
\expandafter\def\csname PY@tok@nf\endcsname{\def\PY@tc##1{\textcolor[rgb]{0.00,0.00,1.00}{##1}}}
\expandafter\def\csname PY@tok@nc\endcsname{\let\PY@bf=\textbf\def\PY@tc##1{\textcolor[rgb]{0.00,0.00,1.00}{##1}}}
\expandafter\def\csname PY@tok@nn\endcsname{\let\PY@bf=\textbf\def\PY@tc##1{\textcolor[rgb]{0.00,0.00,1.00}{##1}}}
\expandafter\def\csname PY@tok@ne\endcsname{\let\PY@bf=\textbf\def\PY@tc##1{\textcolor[rgb]{0.82,0.25,0.23}{##1}}}
\expandafter\def\csname PY@tok@nv\endcsname{\def\PY@tc##1{\textcolor[rgb]{0.10,0.09,0.49}{##1}}}
\expandafter\def\csname PY@tok@no\endcsname{\def\PY@tc##1{\textcolor[rgb]{0.53,0.00,0.00}{##1}}}
\expandafter\def\csname PY@tok@nl\endcsname{\def\PY@tc##1{\textcolor[rgb]{0.63,0.63,0.00}{##1}}}
\expandafter\def\csname PY@tok@ni\endcsname{\let\PY@bf=\textbf\def\PY@tc##1{\textcolor[rgb]{0.60,0.60,0.60}{##1}}}
\expandafter\def\csname PY@tok@na\endcsname{\def\PY@tc##1{\textcolor[rgb]{0.49,0.56,0.16}{##1}}}
\expandafter\def\csname PY@tok@nt\endcsname{\let\PY@bf=\textbf\def\PY@tc##1{\textcolor[rgb]{0.00,0.50,0.00}{##1}}}
\expandafter\def\csname PY@tok@nd\endcsname{\def\PY@tc##1{\textcolor[rgb]{0.67,0.13,1.00}{##1}}}
\expandafter\def\csname PY@tok@s\endcsname{\def\PY@tc##1{\textcolor[rgb]{0.73,0.13,0.13}{##1}}}
\expandafter\def\csname PY@tok@sd\endcsname{\let\PY@it=\textit\def\PY@tc##1{\textcolor[rgb]{0.73,0.13,0.13}{##1}}}
\expandafter\def\csname PY@tok@si\endcsname{\let\PY@bf=\textbf\def\PY@tc##1{\textcolor[rgb]{0.73,0.40,0.53}{##1}}}
\expandafter\def\csname PY@tok@se\endcsname{\let\PY@bf=\textbf\def\PY@tc##1{\textcolor[rgb]{0.73,0.40,0.13}{##1}}}
\expandafter\def\csname PY@tok@sr\endcsname{\def\PY@tc##1{\textcolor[rgb]{0.73,0.40,0.53}{##1}}}
\expandafter\def\csname PY@tok@ss\endcsname{\def\PY@tc##1{\textcolor[rgb]{0.10,0.09,0.49}{##1}}}
\expandafter\def\csname PY@tok@sx\endcsname{\def\PY@tc##1{\textcolor[rgb]{0.00,0.50,0.00}{##1}}}
\expandafter\def\csname PY@tok@m\endcsname{\def\PY@tc##1{\textcolor[rgb]{0.40,0.40,0.40}{##1}}}
\expandafter\def\csname PY@tok@gh\endcsname{\let\PY@bf=\textbf\def\PY@tc##1{\textcolor[rgb]{0.00,0.00,0.50}{##1}}}
\expandafter\def\csname PY@tok@gu\endcsname{\let\PY@bf=\textbf\def\PY@tc##1{\textcolor[rgb]{0.50,0.00,0.50}{##1}}}
\expandafter\def\csname PY@tok@gd\endcsname{\def\PY@tc##1{\textcolor[rgb]{0.63,0.00,0.00}{##1}}}
\expandafter\def\csname PY@tok@gi\endcsname{\def\PY@tc##1{\textcolor[rgb]{0.00,0.63,0.00}{##1}}}
\expandafter\def\csname PY@tok@gr\endcsname{\def\PY@tc##1{\textcolor[rgb]{1.00,0.00,0.00}{##1}}}
\expandafter\def\csname PY@tok@ge\endcsname{\let\PY@it=\textit}
\expandafter\def\csname PY@tok@gs\endcsname{\let\PY@bf=\textbf}
\expandafter\def\csname PY@tok@gp\endcsname{\let\PY@bf=\textbf\def\PY@tc##1{\textcolor[rgb]{0.00,0.00,0.50}{##1}}}
\expandafter\def\csname PY@tok@go\endcsname{\def\PY@tc##1{\textcolor[rgb]{0.53,0.53,0.53}{##1}}}
\expandafter\def\csname PY@tok@gt\endcsname{\def\PY@tc##1{\textcolor[rgb]{0.00,0.27,0.87}{##1}}}
\expandafter\def\csname PY@tok@err\endcsname{\def\PY@bc##1{\setlength{\fboxsep}{0pt}\fcolorbox[rgb]{1.00,0.00,0.00}{1,1,1}{\strut ##1}}}
\expandafter\def\csname PY@tok@kc\endcsname{\let\PY@bf=\textbf\def\PY@tc##1{\textcolor[rgb]{0.00,0.50,0.00}{##1}}}
\expandafter\def\csname PY@tok@kd\endcsname{\let\PY@bf=\textbf\def\PY@tc##1{\textcolor[rgb]{0.00,0.50,0.00}{##1}}}
\expandafter\def\csname PY@tok@kn\endcsname{\let\PY@bf=\textbf\def\PY@tc##1{\textcolor[rgb]{0.00,0.50,0.00}{##1}}}
\expandafter\def\csname PY@tok@kr\endcsname{\let\PY@bf=\textbf\def\PY@tc##1{\textcolor[rgb]{0.00,0.50,0.00}{##1}}}
\expandafter\def\csname PY@tok@bp\endcsname{\def\PY@tc##1{\textcolor[rgb]{0.00,0.50,0.00}{##1}}}
\expandafter\def\csname PY@tok@fm\endcsname{\def\PY@tc##1{\textcolor[rgb]{0.00,0.00,1.00}{##1}}}
\expandafter\def\csname PY@tok@vc\endcsname{\def\PY@tc##1{\textcolor[rgb]{0.10,0.09,0.49}{##1}}}
\expandafter\def\csname PY@tok@vg\endcsname{\def\PY@tc##1{\textcolor[rgb]{0.10,0.09,0.49}{##1}}}
\expandafter\def\csname PY@tok@vi\endcsname{\def\PY@tc##1{\textcolor[rgb]{0.10,0.09,0.49}{##1}}}
\expandafter\def\csname PY@tok@vm\endcsname{\def\PY@tc##1{\textcolor[rgb]{0.10,0.09,0.49}{##1}}}
\expandafter\def\csname PY@tok@sa\endcsname{\def\PY@tc##1{\textcolor[rgb]{0.73,0.13,0.13}{##1}}}
\expandafter\def\csname PY@tok@sb\endcsname{\def\PY@tc##1{\textcolor[rgb]{0.73,0.13,0.13}{##1}}}
\expandafter\def\csname PY@tok@sc\endcsname{\def\PY@tc##1{\textcolor[rgb]{0.73,0.13,0.13}{##1}}}
\expandafter\def\csname PY@tok@dl\endcsname{\def\PY@tc##1{\textcolor[rgb]{0.73,0.13,0.13}{##1}}}
\expandafter\def\csname PY@tok@s2\endcsname{\def\PY@tc##1{\textcolor[rgb]{0.73,0.13,0.13}{##1}}}
\expandafter\def\csname PY@tok@sh\endcsname{\def\PY@tc##1{\textcolor[rgb]{0.73,0.13,0.13}{##1}}}
\expandafter\def\csname PY@tok@s1\endcsname{\def\PY@tc##1{\textcolor[rgb]{0.73,0.13,0.13}{##1}}}
\expandafter\def\csname PY@tok@mb\endcsname{\def\PY@tc##1{\textcolor[rgb]{0.40,0.40,0.40}{##1}}}
\expandafter\def\csname PY@tok@mf\endcsname{\def\PY@tc##1{\textcolor[rgb]{0.40,0.40,0.40}{##1}}}
\expandafter\def\csname PY@tok@mh\endcsname{\def\PY@tc##1{\textcolor[rgb]{0.40,0.40,0.40}{##1}}}
\expandafter\def\csname PY@tok@mi\endcsname{\def\PY@tc##1{\textcolor[rgb]{0.40,0.40,0.40}{##1}}}
\expandafter\def\csname PY@tok@il\endcsname{\def\PY@tc##1{\textcolor[rgb]{0.40,0.40,0.40}{##1}}}
\expandafter\def\csname PY@tok@mo\endcsname{\def\PY@tc##1{\textcolor[rgb]{0.40,0.40,0.40}{##1}}}
\expandafter\def\csname PY@tok@ch\endcsname{\let\PY@it=\textit\def\PY@tc##1{\textcolor[rgb]{0.25,0.50,0.50}{##1}}}
\expandafter\def\csname PY@tok@cm\endcsname{\let\PY@it=\textit\def\PY@tc##1{\textcolor[rgb]{0.25,0.50,0.50}{##1}}}
\expandafter\def\csname PY@tok@cpf\endcsname{\let\PY@it=\textit\def\PY@tc##1{\textcolor[rgb]{0.25,0.50,0.50}{##1}}}
\expandafter\def\csname PY@tok@c1\endcsname{\let\PY@it=\textit\def\PY@tc##1{\textcolor[rgb]{0.25,0.50,0.50}{##1}}}
\expandafter\def\csname PY@tok@cs\endcsname{\let\PY@it=\textit\def\PY@tc##1{\textcolor[rgb]{0.25,0.50,0.50}{##1}}}

\def\PYZbs{\char`\\}
\def\PYZus{\char`\_}
\def\PYZob{\char`\{}
\def\PYZcb{\char`\}}
\def\PYZca{\char`\^}
\def\PYZam{\char`\&}
\def\PYZlt{\char`\<}
\def\PYZgt{\char`\>}
\def\PYZsh{\char`\#}
\def\PYZpc{\char`\%}
\def\PYZdl{\char`\$}
\def\PYZhy{\char`\-}
\def\PYZsq{\char`\'}
\def\PYZdq{\char`\"}
\def\PYZti{\char`\~}
% for compatibility with earlier versions
\def\PYZat{@}
\def\PYZlb{[}
\def\PYZrb{]}
\makeatother


    % For linebreaks inside Verbatim environment from package fancyvrb. 
    \makeatletter
        \newbox\Wrappedcontinuationbox 
        \newbox\Wrappedvisiblespacebox 
        \newcommand*\Wrappedvisiblespace {\textcolor{red}{\textvisiblespace}} 
        \newcommand*\Wrappedcontinuationsymbol {\textcolor{red}{\llap{\tiny$\m@th\hookrightarrow$}}} 
        \newcommand*\Wrappedcontinuationindent {3ex } 
        \newcommand*\Wrappedafterbreak {\kern\Wrappedcontinuationindent\copy\Wrappedcontinuationbox} 
        % Take advantage of the already applied Pygments mark-up to insert 
        % potential linebreaks for TeX processing. 
        %        {, <, #, %, $, ' and ": go to next line. 
        %        _, }, ^, &, >, - and ~: stay at end of broken line. 
        % Use of \textquotesingle for straight quote. 
        \newcommand*\Wrappedbreaksatspecials {% 
            \def\PYGZus{\discretionary{\char`\_}{\Wrappedafterbreak}{\char`\_}}% 
            \def\PYGZob{\discretionary{}{\Wrappedafterbreak\char`\{}{\char`\{}}% 
            \def\PYGZcb{\discretionary{\char`\}}{\Wrappedafterbreak}{\char`\}}}% 
            \def\PYGZca{\discretionary{\char`\^}{\Wrappedafterbreak}{\char`\^}}% 
            \def\PYGZam{\discretionary{\char`\&}{\Wrappedafterbreak}{\char`\&}}% 
            \def\PYGZlt{\discretionary{}{\Wrappedafterbreak\char`\<}{\char`\<}}% 
            \def\PYGZgt{\discretionary{\char`\>}{\Wrappedafterbreak}{\char`\>}}% 
            \def\PYGZsh{\discretionary{}{\Wrappedafterbreak\char`\#}{\char`\#}}% 
            \def\PYGZpc{\discretionary{}{\Wrappedafterbreak\char`\%}{\char`\%}}% 
            \def\PYGZdl{\discretionary{}{\Wrappedafterbreak\char`\$}{\char`\$}}% 
            \def\PYGZhy{\discretionary{\char`\-}{\Wrappedafterbreak}{\char`\-}}% 
            \def\PYGZsq{\discretionary{}{\Wrappedafterbreak\textquotesingle}{\textquotesingle}}% 
            \def\PYGZdq{\discretionary{}{\Wrappedafterbreak\char`\"}{\char`\"}}% 
            \def\PYGZti{\discretionary{\char`\~}{\Wrappedafterbreak}{\char`\~}}% 
        } 
        % Some characters . , ; ? ! / are not pygmentized. 
        % This macro makes them "active" and they will insert potential linebreaks 
        \newcommand*\Wrappedbreaksatpunct {% 
            \lccode`\~`\.\lowercase{\def~}{\discretionary{\hbox{\char`\.}}{\Wrappedafterbreak}{\hbox{\char`\.}}}% 
            \lccode`\~`\,\lowercase{\def~}{\discretionary{\hbox{\char`\,}}{\Wrappedafterbreak}{\hbox{\char`\,}}}% 
            \lccode`\~`\;\lowercase{\def~}{\discretionary{\hbox{\char`\;}}{\Wrappedafterbreak}{\hbox{\char`\;}}}% 
            \lccode`\~`\:\lowercase{\def~}{\discretionary{\hbox{\char`\:}}{\Wrappedafterbreak}{\hbox{\char`\:}}}% 
            \lccode`\~`\?\lowercase{\def~}{\discretionary{\hbox{\char`\?}}{\Wrappedafterbreak}{\hbox{\char`\?}}}% 
            \lccode`\~`\!\lowercase{\def~}{\discretionary{\hbox{\char`\!}}{\Wrappedafterbreak}{\hbox{\char`\!}}}% 
            \lccode`\~`\/\lowercase{\def~}{\discretionary{\hbox{\char`\/}}{\Wrappedafterbreak}{\hbox{\char`\/}}}% 
            \catcode`\.\active
            \catcode`\,\active 
            \catcode`\;\active
            \catcode`\:\active
            \catcode`\?\active
            \catcode`\!\active
            \catcode`\/\active 
            \lccode`\~`\~ 	
        }
    \makeatother

    \let\OriginalVerbatim=\Verbatim
    \makeatletter
    \renewcommand{\Verbatim}[1][1]{%
        %\parskip\z@skip
        \sbox\Wrappedcontinuationbox {\Wrappedcontinuationsymbol}%
        \sbox\Wrappedvisiblespacebox {\FV@SetupFont\Wrappedvisiblespace}%
        \def\FancyVerbFormatLine ##1{\hsize\linewidth
            \vtop{\raggedright\hyphenpenalty\z@\exhyphenpenalty\z@
                \doublehyphendemerits\z@\finalhyphendemerits\z@
                \strut ##1\strut}%
        }%
        % If the linebreak is at a space, the latter will be displayed as visible
        % space at end of first line, and a continuation symbol starts next line.
        % Stretch/shrink are however usually zero for typewriter font.
        \def\FV@Space {%
            \nobreak\hskip\z@ plus\fontdimen3\font minus\fontdimen4\font
            \discretionary{\copy\Wrappedvisiblespacebox}{\Wrappedafterbreak}
            {\kern\fontdimen2\font}%
        }%
        
        % Allow breaks at special characters using \PYG... macros.
        \Wrappedbreaksatspecials
        % Breaks at punctuation characters . , ; ? ! and / need catcode=\active 	
        \OriginalVerbatim[#1,codes*=\Wrappedbreaksatpunct]%
    }
    \makeatother

    % Exact colors from NB
    \definecolor{incolor}{HTML}{303F9F}
    \definecolor{outcolor}{HTML}{D84315}
    \definecolor{cellborder}{HTML}{CFCFCF}
    \definecolor{cellbackground}{HTML}{F7F7F7}
    
    % prompt
    \makeatletter
    \newcommand{\boxspacing}{\kern\kvtcb@left@rule\kern\kvtcb@boxsep}
    \makeatother
    \newcommand{\prompt}[4]{
        \ttfamily\llap{{\color{#2}[#3]:\hspace{3pt}#4}}\vspace{-\baselineskip}
    }
    

    
    % Prevent overflowing lines due to hard-to-break entities
    \sloppy 
    % Setup hyperref package
    \hypersetup{
      breaklinks=true,  % so long urls are correctly broken across lines
      colorlinks=true,
      urlcolor=urlcolor,
      linkcolor=linkcolor,
      citecolor=citecolor,
      }
    % Slightly bigger margins than the latex defaults
    
    \geometry{verbose,tmargin=1in,bmargin=1in,lmargin=1in,rmargin=1in}
    
    

\begin{document}
    
    \maketitle
    
    

    
    \hypertarget{github-exercise}{%
\section{GitHub exercise}\label{github-exercise}}

The final part of the workshop focuses on implementing a software
package for calculating different distance measures. The exercise uses
the hosting service GitHub.

This part of the workshop covers:

\begin{itemize}
\tightlist
\item
  Creating GitHub repositories
\item
  Packaging and testing software
\item
  Creating pull request
\item
  Lincense
\item
  Package documentation
\item
  Continuous integration
\end{itemize}

    GitHub is one hosting service but several other exists, for example:

\begin{itemize}
\tightlist
\item
  GitLab: https://about.gitlab.com
\end{itemize}

    \hypertarget{log-in-github}{%
\section{Log in GitHub}\label{log-in-github}}

Initially make sure that you are logged in your GitHub account:
https://github.com.

    \hypertarget{create-a-repository-for-the-project}{%
\section{Create a repository for the
project}\label{create-a-repository-for-the-project}}

    A software repository, or ``repo'' for short, is a storage location for
software packages.

To create a repository on GitHub click on ``New'' located at the left
side of the home page.

    Here we are going to create the project's repository.

\begin{itemize}
\tightlist
\item
  \textbf{Repository name}: distances
\item
  \textbf{Description}: A package for calculating different distance
  measures.
\item
  \textbf{Private}. We are going to create a private repository as this
  is an artificial exercise.
\item
  \textbf{Initialize this repository}. GitHub allows us to initialize a
  repository with a series of standard files. We want to include:

  \begin{itemize}
  \tightlist
  \item
    A README. A README file contains information about other files in a
    directory or archive of computer software.
  \item
    A .gitignore. Select the templete Python
  \item
    A lincense. Select the MIT License.
  \end{itemize}
\end{itemize}

    

    We click \texttt{create\ repository} and this creates our repository.

    \hypertarget{clone-the-repository}{%
\section{Clone the repository}\label{clone-the-repository}}

When we create a repository on GitHub, we create a project that exists
as a remote repository. We can clone our repositories to create a local
copy on our computers and sync between the two locations.

To clone a GitHub repository we need to copy it's address:

and run the following command in a terminal/command prompt:

\begin{verbatim}
$ git clone <the_address_we_just_copied>
\end{verbatim}

Make sure that you run this command while on the location you want to
clone the repository.

For example if I wanted to clone the repository on my Deskop I would run
the following commands:

\begin{verbatim}
$ cd Desktop
$ git clone https://github.com/Nikoleta-v3/distances.git
\end{verbatim}

    \hypertarget{brach}{%
\section{Brach}\label{brach}}

When working on a project we hardly ever work on the \texttt{master}
branch. Instead we create different branches for working on different
parts of a project.

So let's create a branch called \texttt{implement-distances-package}.

\begin{verbatim}
$ git branch implement-distances-package
$ git checkout implement-distances-package
\end{verbatim}

    The repository's structure right now is as follows:

\begin{verbatim}
|--- .gitingore
|--- LICENSE   
|--- README.md
\end{verbatim}

Spend some time now to familirie yourself with the structure and the
current files of the repository.

    \hypertarget{implement-the-distance-function}{%
\section{Implement the distance
function}\label{implement-the-distance-function}}

We are now ready to write a package that calculates distances.

\textbf{1.} We will create a folder called \texttt{distances}. This can
be done with the command:

\begin{verbatim}
$ mkdir distances
\end{verbatim}

\textbf{2.} In the folder we just created we are going to add a file
called \texttt{euclidean.py}. This file will contain the code needed to
calculate the euclidean distance of two vectors.

Alter the file \texttt{euclidean.py} to contain the following lines of
code:

\begin{Shaded}
\begin{Highlighting}[]
\ImportTok{import}\NormalTok{ math }

\KeywordTok{def}\NormalTok{ euclidean\_distance(u, v):}
    \CommentTok{"""}
\CommentTok{    Computes the Euclidean distance between two vectos \textasciigrave{}u\textasciigrave{} and \textasciigrave{}v\textasciigrave{}.}

\CommentTok{    The Euclidean distance between \textasciigrave{}u\textasciigrave{} and \textasciigrave{}v\textasciigrave{}, is defined as:}

\CommentTok{    \textbackslash{}sqrt\{(u\_1 {-} v\_1) \^{} 2 + ... + (u\_n {-} v\_n) \^{} 2\}}

\CommentTok{    Parameters}
\CommentTok{    {-}{-}{-}{-}{-}{-}{-}{-}{-}{-}}
\CommentTok{    u : list}
\CommentTok{        Input vector.}
\CommentTok{    v : list}
\CommentTok{        Input vector.}

\CommentTok{    Returns}
\CommentTok{    {-}{-}{-}{-}{-}{-}{-}}
\CommentTok{    euclidean : double}
\CommentTok{        The Euclidean distance between vectors \textasciigrave{}u\textasciigrave{} and \textasciigrave{}v\textasciigrave{}.}
\CommentTok{    """}
\NormalTok{    distance }\OperatorTok{=} \DecValTok{0}
    
    \ControlFlowTok{for}\NormalTok{ u\_i, v\_i }\KeywordTok{in} \BuiltInTok{zip}\NormalTok{(u, v):}
\NormalTok{        distance }\OperatorTok{+=}\NormalTok{ (u\_i }\OperatorTok{{-}}\NormalTok{ v\_i) }\OperatorTok{**} \DecValTok{2}
        
    \ControlFlowTok{return}\NormalTok{ math.sqrt(distance)}
\end{Highlighting}
\end{Shaded}

\textbf{3.} Now that the function is implemented we need to commit the
change.

\begin{verbatim}
$ git add distances/euclidean.py
$ git commit
\end{verbatim}

You can use the following commit message:

\begin{verbatim}
implement euclidean distance
\end{verbatim}

    The structure of the repository should now be the following:

\begin{verbatim}
|--- distances
    |--- euclidean.py
|--- .gitingore
|--- LICENSE   
|--- README.md
\end{verbatim}

    \hypertarget{test-the-function}{%
\section{Test the function}\label{test-the-function}}

    Now that we have written the function it is time to use it. Creating a
test for an implemented function is a great way to:

\begin{enumerate}
\def\labelenumi{\arabic{enumi}.}
\tightlist
\item
  Demonstrate it's usage
\item
  Test it's implementation
\end{enumerate}

Lets's create a file \texttt{test\_euclidean.py} at the root of the
repository such as the structure now is:

\begin{verbatim}
|--- distances
    |--- euclidean.py
|--- .gitingore
|--- LICENSE   
|--- README.md
|--- test_euclidean.py
\end{verbatim}

Open \texttt{test\_euclidean.py} with your editor and alter it so that
it looks like:

\begin{Shaded}
\begin{Highlighting}[]
\ImportTok{import}\NormalTok{ distances}

\NormalTok{u }\OperatorTok{=}\NormalTok{ (}\DecValTok{2}\NormalTok{, }\OperatorTok{{-}}\DecValTok{1}\NormalTok{)}
\NormalTok{v }\OperatorTok{=}\NormalTok{ (}\OperatorTok{{-}}\DecValTok{2}\NormalTok{, }\DecValTok{2}\NormalTok{)}

\BuiltInTok{print}\NormalTok{(distances.euclidean\_distance(u, v))}
\end{Highlighting}
\end{Shaded}

Trying to run this file using the command:

\begin{verbatim}
$ python test_euclidean.py
\end{verbatim}

should return the following error:

In order to let python know that files under the module
\texttt{distances} we need to creat a file called
\texttt{\_\_init\_\_.py} which will be under \texttt{distances} and will
include the following line:

\begin{Shaded}
\begin{Highlighting}[]
\ImportTok{from}\NormalTok{ .euclidean }\ImportTok{import}\NormalTok{ euclidean\_distance}
\end{Highlighting}
\end{Shaded}

Now run the command:

\begin{verbatim}
$ python test_euclidean.py
\end{verbatim}

again.

Now let's alter the code to include an \texttt{assert} command:

\begin{Shaded}
\begin{Highlighting}[]
\ImportTok{import}\NormalTok{ distances}

\NormalTok{u }\OperatorTok{=}\NormalTok{ (}\DecValTok{2}\NormalTok{, }\OperatorTok{{-}}\DecValTok{1}\NormalTok{)}
\NormalTok{v }\OperatorTok{=}\NormalTok{ (}\OperatorTok{{-}}\DecValTok{2}\NormalTok{, }\DecValTok{2}\NormalTok{)}

\ControlFlowTok{assert}\NormalTok{ distances.euclidean\_distance(u, v) }\OperatorTok{==} \DecValTok{5}
\end{Highlighting}
\end{Shaded}

and run \texttt{python\ test\_euclidean.py}.

Fantastic. Now, let's commit this change:

\begin{verbatim}
$ git add test_euclidean.py
$ git add distances/__init__.py
\end{verbatim}

with the following commit:

\begin{verbatim}
add test for the euclidean distance
\end{verbatim}

    \hypertarget{pytest}{%
\section{Pytest}\label{pytest}}

Currently the command:

\begin{verbatim}
$ python test_euclidean.py
\end{verbatim}

does not give any feedback. In our case that is because the
\texttt{assert} command is \texttt{True} and thus there is nothing to
report.

Several packages already exist that can make tests output more useful.
An example of such a Python package is \texttt{pytest}. \texttt{pytest}
has been installed on your computers with Anaconda.

Alter the code in \texttt{test\_euclidean.py} to:

\begin{Shaded}
\begin{Highlighting}[]
\ImportTok{import}\NormalTok{ distances}

\KeywordTok{def}\NormalTok{ test\_euclidean():}
\NormalTok{    u }\OperatorTok{=}\NormalTok{ (}\DecValTok{2}\NormalTok{, }\OperatorTok{{-}}\DecValTok{1}\NormalTok{)}
\NormalTok{    v }\OperatorTok{=}\NormalTok{ (}\OperatorTok{{-}}\DecValTok{2}\NormalTok{, }\DecValTok{2}\NormalTok{)}

    \ControlFlowTok{assert}\NormalTok{ distances.euclidean\_distance(u, v) }\OperatorTok{==} \DecValTok{5}
\end{Highlighting}
\end{Shaded}

and now use the following command to run the tests:

\begin{verbatim}
$ pytest test_euclidean.py
\end{verbatim}

This should return something like:

Now commit the changes with a message:
\texttt{use\ the\ library\ pytest\ to\ run\ tests}.

The commands are:

\begin{verbatim}
$ git add test_euclidean.py
$ git commit
\end{verbatim}

    \hypertarget{document-the-package}{%
\section{Document the package}\label{document-the-package}}

When you are developing a software package, you want to make available
for other people to use. Thus, you need to document your project.

This include, letting people know how to install your project, its
purpose and functionality, its license and how to test it.

All these details can be included on a projects \texttt{README.md}.

The current \texttt{README.md} of our project looks like this:

\begin{quote}
\hypertarget{distances}{%
\section{distances}\label{distances}}

A package for calculating different distance measures.
\end{quote}

Alter the file to include details of your project.

An example of a \texttt{README.md}:

\begin{quote}
\hypertarget{distances-1}{%
\section{distances}\label{distances-1}}

A package for calculating different distance measures. \# Installation
The project can be cloned locally using the following command:
\end{quote}

\begin{quote}
\begin{verbatim}
$ git clone <path>
\end{verbatim}
\end{quote}

\begin{quote}
\hypertarget{usage}{%
\section{Usage}\label{usage}}

Currently the following distance measures are implemented in the
package: - Euclidean distance. \# Tests

The package is being tested using \texttt{pytest}. To run the test suite
use the command:

\begin{verbatim}
$ pytest test_euclidean
\end{verbatim}

\hypertarget{license}{%
\section{License}\label{license}}

The package is under the MIT license.
\end{quote}

\textbf{Once you are done altering the \texttt{README.md} remember to
commit your changes.}

    \hypertarget{open-a-pull-request}{%
\section{Open a pull request}\label{open-a-pull-request}}

    Our project is now ready. We have implemented a package that can
calculate the euclidean distance and we want to share it with the world.

Before we update our copy on GitHub run the command:

\begin{verbatim}
$ git status
\end{verbatim}

We can see that there are a few files that have not been committed, but
also that we don't reconsiged these files. You don't need to worry about
these files. Let's add them to our \texttt{.gitignore}.

So our \texttt{.gitingore} should include the lines:

\begin{verbatim}
__pycache__/
distances/__pycache__/
\end{verbatim}

Add the changes, commit and run \texttt{git\ status} again.

    Now that everything has been committed we are ready to update the copy
of our project on GitHub.

To do that you need to run the following command:

\begin{verbatim}
$ git push origin implement-distances-package
\end{verbatim}

and then you should something like this:

    Once you have run the command open GitHub and navigate your project.

There you should see the following:

Click on \texttt{Compare\ \&\ pull\ request}.

GitHub then transfers you to \texttt{Pull\ request}. Here you can review
the changes you have made and \texttt{request} for your code to become
part of the projects main branch.

Take sometime to familirize yourself with the \texttt{Pull\ request}
page and then click on \texttt{Create\ pull\ request}.

Congratulations you have created your first pull request!!!!!!

    \begin{tcolorbox}[breakable, size=fbox, boxrule=1pt, pad at break*=1mm,colback=cellbackground, colframe=cellborder]
\prompt{In}{incolor}{ }{\boxspacing}
\begin{Verbatim}[commandchars=\\\{\}]

\end{Verbatim}
\end{tcolorbox}


    % Add a bibliography block to the postdoc
    
    
    
\end{document}
